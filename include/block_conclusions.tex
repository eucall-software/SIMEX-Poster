\block[bodyoffsetx=0pt]{Conclusions}{%
  \begin{itemize}
    \item Diffraction data from hydrated proteins give estimate of tolerable
      water layer thickness
    \item[$\Rightarrow$] $R < 0.2$ criterion indicates that \SI{3}{\angstrom} thick water layer permits
      imaging at few \si{\angstrom} resolution.
    \item The coefficient of variation
      $\sigma_\text{V}(k) = \Bigg\langle
        \frac{
          \sqrt{
            \langle I(\mathbf{k})^{2}\rangle - \langle I(\mathbf{k})\rangle^{2}
          }
        }{
          \langle I(\mathbf{k})\rangle
        }
        \Bigg\rangle_{\mathbf{k}: |\mathbf{k}| = k}$
         quantifies the ``interpretability'' of simulated diffraction data under real--world conditions.
    \item Extremely short (\SI{3}{\femto\second}) XFEL pulses outrun radiation damage in
      single--protein imaging experiments at SPB--SFX.
    \item Lower pulse fluence at \SI{3}{\femto\second} yields
      low S/N and high $\sigma_\text{V}(k)$ in
      oriented 3D diffraction data.
    \item[$\Rightarrow$] Optimal pulse duration for small proteins closer to
      \SI{9}{\femto\second} despite radiation damage.
  \end{itemize}
}%
